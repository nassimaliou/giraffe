
\chapter{Summary of the original paper draven}
    \section{Probabilistic search}
        \subsection{Alternative formulation of the problem}
        
            Instead of trying to construct the theoritical proncipal variation --which necessitates a complete traversal of the game tree--, the authors propose the goal of ``\emph{finding as much of it as possible}''.

        \subsection{Probability-limited search}

            In contrast to the more conventional \emph{depth-limited search} where one explores all nodes up to a certain depth (in this case all possible games with a certain number of moves), probablity-limited search looks for all nodes with a certain probability of being part of the theoreitical PV of the root position.
            
            At each node, we compute the probability of its descendents being in the PV given that the node itself is. We start at the root --which has a \(100\%\) propability of being in the PV--, and we propagate the probabilities recursively downwards.
            
            Instead of the halting condition beeing the reaching of a maximum depth, we stop when the probability drops below a certain threshold. We then evaluate the current node and return the result from the \verb|minimax| call.