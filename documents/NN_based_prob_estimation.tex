\section{Neural Network-Based Probability Estimation}
		\subsection{Neural Network-Based Probability Estimation}
			Can be resumed into two parts:
			\begin{enumerate}
				\item Describing the parent position
				\item Describing the move under consideration
			\end{enumerate}
			The part that describes the parent position is exactly the same as the feature representation we use for position evaluation, as described in the feature representation of Neural Network-Based Evaluation.With additional features as follows:
			\begin{enumerate}
				\item Piece Type
				\item From Square
				\item To Square
				\item Promotion Type
				\item Rank
			\end{enumerate}
		\subsection{Network Architecture}
			Similar to the architecture for the evaluation network described in network architecture of Neural Network-Based Evaluation .The only difference is that the output node has logistic activation instead of hyperbolic tangent activation. It also has rectified linear activation for hidden nodes, and separately-connected layers for features from the three different modalities.
		\subsection{Training Positions Generation}
		We need a corpus of unlabeled positions.
		But not the same corpus used in Evaluation network training.
		For training the evaluation network using TD-Leaf($\lambda$), we need the training positions to be representative of root positions that will appear in actual games.
		To generate this training set :
		\begin{enumerate}
			\item Performing time-limited searches on the root positions from the training set for evaluation network training
			\item Performing a random sample from the positions encountered as internal nodes in those searches
		\end{enumerate}

		\subsection{Network Training}
			We have a trained evaluator which can find the best move for each training position, as a consequence, we don’t need an iterative temporal-difference learning.
			\begin{enumerate}
				\item \textbf{First step:}
					label each training position with the best move by
					performing a time-limited search on each one of them.
				\item \textbf{Second step:}
					the generation of the training set for gradient descent by combining each position with all legal moves from that position, labeled by a binary training target specifying whether the move is the best move or not.
				\item \textbf{Last step:}
					performing a stochastic gradient descent with AdaDelta update rules using the cross-entropy loss function.
			\end{enumerate}